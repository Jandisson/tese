\chapter{Conclusão}



\section{Trabalhos futuros}

\begin{itemize}
\item Necessidade de complementação da pesquisa por meio de estudos qualitativos. Os resultados obtidos precisam ser explicados.
\item Falar sobre as possibilidades de utilizar o modelo para prever a produtividade futura dos software baseado na estimativa de crescimento da dívida técnica.
\end{itemize}


\subsection{Popularidade vs Dívida Técnica}

Poderia ter um capítulo falando sobre as relações entre a atratividade de um projeto (quantas pessoas procuram colaborar com ele) e a dívida técnica.
A existência de muita dívida técnica afasta novos colaboradores?

Nos dados obtidos dos projetos do piloto é possível notar que os projetos com mais pull requests possuem menos dívida técnica.

Um assunto a ser abordado poderia ser a relação entre a quantidade de forks e a quantidade de pull requests. É possível nota no pilo que alguns projetos tem uma quantidade de forks muito maior do que a quantidade de pull requests. Ou seja, muitas pessoas fizeram uma cópia do projeto com a interação de realizar alterações mas, muitas delas não o fizeram.

\subsection{Artigo CBI}

Pode ser que a relação entre dívida técnica e linhas de código seja inversamente proporcional. Ou seja, quanto mais dívida técnica um projeto tem, mais linhas de código ele tem também. Falar sobre isso seria uma boa oportunidade para inserir o artigo apresentado na CBI e o Jornal.


