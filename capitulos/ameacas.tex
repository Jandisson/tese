\chapter{Ameaças à validade}

Podemos identificar algumas dificuldades em realizar a estimação dos juros utilizando essa estratégia. A seguir destacamos algumas delas:
 
 \begin{itemize}
 \item A dificuldade em si de medir a produtividade no desenvolvimento de software.
 \item Identificar qual a \textbf{produtividade ótima} para um projeto de software. Para isso, utilizaremos analogia. Ou seja, para um determinado projeto iremos identificar projetos semelhantes e com um baixo índice de dívida técnica.
 \item Identificar se um projeto é semelhante a outro. Na literatura existem algumas abordagens para realizar essa atividade. Iremos descrever essas abordagens e explicar como iremos adaptá-las nesse projeto.
 \item Isolar a interferência da dívida técnica nos projetos de software da interferência de outros fatores que estão prejudicando a produtividade. Ou seja, identificar e garantir que o que está fazendo um projeto não estar no cenário de \textbf{produtividade ótima} é exclusivamente a dívida técnica.
 \end{itemize}
 
 Descreveremos cada uma dessas dificuldades e como iremos resolvê-las ou diminuir sua interferência nos resultados desta pesquisa.
 
 
 Preciso falar sobre o problema de empresas famosas como NetFlix possuirem um alto número de estrelas. Isso também acontece com profissionais que trabalham nessas empresas. Normalmente, eles também terão um grande número de seguidores. Isso obviamente nem sempre está relacionado à expertise dessas pessoas. Ao invés disso, eles podem ter um número alto de seguidores pois as pessoas normalmente se interessam pelas novidades dos projetos criados por essas empresas. 
 
 \section{Escala}

A literatura sugere que projetos maiores vão ter um prejuizo de produtividade devido à complexidade adicional trazida pelo seu tamanho. Realizar alterações em um projeto menor normalmente é mais fácil do que realizar alterações em projetos menores. Isso deveria ser considerado em um modelo como o nosso que tem como base uma estimação de produtividade.

\section{Avaliação empírica das métricas utilizadas}

Muitas métricas que utilizamos na pesquisa poderiam ser validas por meio de estudos empíricos. Um exemplo é o modelo de agregação utilizado para representar a colaboração nos projetos. Foram utilizados três aspectos: quantidade, assiduidade e qualidade. Entretanto, esse modelo não foi validade em nenhum momento. Não é possível garantir que esse modelo é adequado e realmente captura o nível de colaboração que um projeto teve.


\section{Coisas a serem feitas caso haja tempo disponível}

\begin{itemize}
\item Realizar um esforço maior para defender a utilização das linhas de código como uma métrica de tamanho do software. Posso buscar mais na literatura trabalhos que mostrem empiricamente o valor dessa métrica.
\end{itemize}